\documentclass{../industrial-development}
\graphicspath{{14-leadership-and-development-team-management/}}

\title{Лидерство руководителя и управление командой разработчиков}
\author{Петров Алексей Анатольевич, ИВТ-21 МО}
\date{}

\begin{document}

\begin{frame}
  \titlepage
\end{frame}

\section{Лидерство руководителя и управление командой разработчиков ПО}

\begin{frame} \frametitle{Понятие лидерства}
 Лидер --- это человек, который может влиять на поведение других людей, брать на себя ответственность, последовательно идти к достижению конкретных целей и вести за собой команду.
\end{frame}

\lecturenotes

Лидерство --- способность оказывать влияние как на~отдельную личность, так и на группы, направляя усилия всех на достижение целей организации. В переводе с английского лидер означает «руководитель», «командир», «глава», «вождь», «ведущий». Группа, решающая значимую проблему, всегда выдвигает для ее решения лидера. Без лидера ни одна группа существовать не может. Лидера можно определить как личность, способную объединять людей ради достижения какой-либо цели. Понятие «лидер» приобретает значение лишь вместе с понятием «цель». Действительно, нелепо бы выглядел лидер, не имеющий цели.
Но иметь цель и достичь ее самостоятельно, в одиночку -- недостаточно, чтобы назваться лидером. Неотъемлемым свойством лидера является наличие хотя бы одного последователя. Роль лидера заключается в умении повести людей за собой, обеспечить существование таких связей между людьми в системе, которые способствовали бы решению конкретных задач в рамках единой цели. Т. е. лидер --- это элемент упорядочивания системы людей. лидерство персонал потенциал.


\begin{frame} \frametitle{Качества лидера}
  \begin{itemize}
  \item Инициативность
  \item Дипломатичность
  \item Умение создать команду
  \item Аналитика
  \item Стратегическое мышление
  \item Уверенность в себе
  \item Готовность к риску  
  \end{itemize}
\end{frame}

\lecturenotes
Инициативность
Инициативные и активные люди не ждут, пока кто-то другой вызовет в них желание работать. Они знают, что только на них лежит ответственность убедить себя оставить привычную зону комфортности. И они делают самомотивацию регулярной практикой. 

Дипломатичность: для того чтобы успешно разрешать возникающие конфликты и всякого рода противоречия. Без этого качества также не обойтись в ситуации, когда тим лиду надо указать человеку на недостатки в его работе.

Умение создать команду
Умение притягивать к себе людей (идеями и мыслями, идеалами, умением убедить человека) и создавать команду последователей и единомышленников – качество, которое определяет успешность становления индивида в роли лидера. Постановка  общих целей и ценностей и контроль над приверженностью последователей этим целям являются одними из главных лидерских качеств.

Аналитик: для того, чтобы адекватно оценивать сроки, риски , возможные проблемы и запасные варианты на случай их возникновения

Стратегическое мышление
Стратегическое мышление помогает в достижении поставленных целей, решении сложных задач, преодолении трудностей. Оно стимулирует человека на борьбу с проблемами качественнее, быстрее, с меньшими затратами на оптимизацию, в то время, как человек с обыденным мышлением более консервативен и действует по шаблону, предпочитая рутинную работу экспериментам.

УВЕРЕННОСТЬ В СЕБЕ
это уверенность в собственных способностях и правоте своих
суждений, решений и идей. Лидер, позитивно оценивающий себя и проявляющий
уверенность в собственных способностях, вызывает доверие, уважение и восхищение
группы. Такой человек своим примером мотивирует окружающих и вдохновляет их на преодоление сложностей.


\begin{frame} \frametitle{Роли лидера в команде}
  \begin{itemize}
  \item Штурман 
  \item Образец 
  \item Помощник 
  \item Вдохновитель 
  \end{itemize}
\end{frame}

\begin{frame} \frametitle{Роли лидера в команде}
\begin{itemize}
\item Штурман — формирует общее видение целей и систему ценностей, определяет курс, учитывая постоянные изменения, которые происходят вокруг и находя новые возможности.
\item Образец для подражания с точки зрения человеческих качеств. Личность, которая заслуживает полное доверие. «Учитель не тот, кто учит, а тот, у которого учатся».
  \end{itemize}
\end{frame}

\begin{frame} \frametitle{Роли лидера в команде}
\begin{itemize}
\item Помощник — создает и, когда необходимо, меняет структуры, процессы, условия, которые обеспечивают эффективность работы каждого. Лидеры следуют правилам до того момента, пока они не увидят, что правила перестают действовать.

\item Вдохновитель — выявляет и направляет способности каждого на достижение результатов, а не на процессы и методы. Поощряет свободу, ответственность, инициативу и творчество, признает право на ошибку
  \end{itemize}
\end{frame}

\lecturenotes
Штурман — формирует общее видение целей и систему ценностей, определяет курс, учитывая постоянные изменения, которые происходят вокруг и находя новые возможности.
Образец для подражания с точки зрения человеческих качеств. Личность, которая заслуживает полное доверие. «Учитель не тот, кто учит, а тот, у которого учатся».
Помощник — создает и, когда необходимо, меняет структуры, процессы, условия, которые обеспечивают эффективность работы каждого. Лидеры следуют правилам до того момента, пока они не увидят, что правила перестают действовать.
Вдохновитель — выявляет и направляет способности каждого на достижение результатов, а не на процессы и методы. Поощряет свободу, ответственность, инициативу и творчество, признает право на ошибку.

\section{Понятие лидерства и отличие между руководителем, и лидером.}

\begin{frame} \frametitle{Типы лидерства}
 	  \begin{block}{Формальный лидер}
           Управляет людьми согласно действующим положениям и~инструкциям.
	  \end{block}

	  \begin{block}{Харизматический лидер}
          Получивший власть над людьми естественным путем благодаря своим личностным качествам.
	  \end{block}
\end{frame}

\lecturenotes

Лидерство  и управление не следует противопоставлять друг другу. Это две составляющих одного и того же процесса. Лидерство — это высшее проявление менеджмента. Современные менеджеры должны быть и управленцами, и лидерами. Лидерство и управление одинаково важны, они не могут существовать в отрыве друг от друга. Нельзя быть лидером запасов, денежных потоков и затрат. Ими необходимо управлять. Потому что у вещей нет права и свободы выбора, которые присущи только человеку. С людьми нужно быть эффективными, а производительными — с вещами. Нельзя ориентироваться на производительность в отношениях с людьми. Интеллектуальными людьми невозможно управлять. Творческие команды можно только направлять и вести.
Управление — это расчленение, анализ, определение последовательности действий, конкретная реализация. Управление фокусируется на нижнем уровне: как мне сделать это наилучшим образом? Эта компетенция руководителя определяет эффективность движения по выбранному пути
Для людей с техническим образованием выполнение роли управленца, как правило не вызывает особых затруднений.
Однако в инновационных проектах, с высокой степенью неопределенности и подверженных высоким рискам просто управления недостаточно. «Высокопроизводительное управление в отсутствие эффективного лидерства подобно упорядочению расстановки стульев на палубе тонущего «Титаника». Никакой успех в управлении не компенсирует провала в лидерстве. ...Лидерство — это высшее проявление менеджмента, это создание для людей перспективы и высвобождение их потенциала». Лидерство — это, прежде всего, энергичная деятельность правого полушария мозга. Оно сродни искусству и философии. 
\begin{frame} \frametitle{Виды лидерства}
  \begin{itemize}
  \item Традиционный лидер
  \item Командный лидер
  \item Репрезентативный лидер
  \item Каталитический лидер
  \end{itemize}
\end{frame}

\lecturenotes

Кроме типов лидерства, можно также выделить несколько основных видов:

Традиционный лидер — получивший власть по наследству, которая узаконена вековыми или ставшими священными традициями. Обычно это монархи, религиозные лидеры, вожди племен. В основе такого лидерства лежит привычка людей к исторически сложившемуся типу власти.
Командный лидер — приходит к власти "силовым путем", умело настроив общество против своих конкурентов. Взяв бразды правления в свои руки, указывает людям, что нужно делать. Использует свои идеи. Действует авторитарными методами, не зависит от чужих мнений, обычно сам создает новое общественное, социальное или политическое движение. Результативен. Если командный лидер переоценивает свои возможности и свое влияние или демонстрирует явные черты деспотизма, это неизбежно приводит к внешнему конфликту: люди выходят из повиновения и восстают против него.
Репрезентативный лидер — наделенный властью теми, чьи пожелания он должен выполнять. Власть ему, по существу, делегирована теми, кого он представляет. Возможности такого лидера ограничены, потому он сдержан, никогда не идет на риск, стараясь действовать наверняка. Работает на основе явно выраженных пожеланий других людей. Личность, как правило, заурядная, лишенная "искры Божьей".
Каталитический лидер — пришедший к власти благодаря тому, что сумел уловить и выразить невысказанные идеи и чаяния группы людей или общества в целом. Обладает тонкой интуицией, острой восприимчивостью, умением распознавать и четко формулировать наметившиеся тенденции развития общества. Начинает действовать до того, как сложится общественное мнение, таким образом ускоряя процесс социальной эволюции, но не изменяя при этом обусловленности ее направления.

\begin{frame} \frametitle{Различия между лидером и руководителем}
Лидерство  и управление не следует противопоставлять друг другу. Лидерство и управление одинаково важны, они не могут существовать в отрыве друг от друга. \\

Управление — это расчленение, анализ, определение последовательности действий, конкретная реализация. Эта компетенция руководителя определяет эффективность движения по выбранному пути
\end{frame}

\lecturenotes

Различия между лидером и руководителем:
1. Руководитель назначается официально, лидер выдвигается неофициально.
2. Руководство выступает как явление более стабильное, чем лидерство. Лидерство является стихийным процессом в отличие от руководства.
3. Руководителю права и полномочия даны законом. Лидер не обладает подобными правами и полномочиями.
4. Руководитель в процессе влияния на подчиненных имеет значительно больше санкций, чем лидер, он может использовать формальные и неформальные санкции. Лидер имеет возможности использовать только неформальные санкции.
5. Руководитель входит в макросреду, его сфера деятельности шире. Лидер является представителем своей группы, ее членом, выступает как элемент микросреды, сфера деятельности лидера ограничивается рамками данной группы.
6. Руководитель регулирует формальные отношения. Деятельность лидера ограничивается рамками межличностных отношений.
7. Для принятия решений руководитель использует большой объем информации, как внешней, так и внутренней. Лидер владеет только той информацией, которая существует в рамках данной группы. Принятие решений руководителем осуществляется опосредованно, а лидером - непосредственно.
8. Руководитель несет внешнюю персональную ответственность за деятельность группы и ее результаты, в том числе перед законом. Лидер не несет подобной ответственности за работу группы и за все, что в ней происходит (если, конечно, группа в своей деятельности не нарушает закон).
9. Руководитель может обладать авторитетом, а может и не иметь его совсем. Лидер всегда авторитетен, в противном случае он не будет лидером.

\section{Специфика команды разработчиков. Роль лидерства.}

\begin{frame} \frametitle{Становление команды и этапы формирования}
 \begin{block}{Этапы формирования команды}
  \begin{itemize}
  \item Объединение
  \item Разногласия и конфликты
  \item Становление
  \item Отдача
  \end{itemize}
  \end{block}
 Успех любого проекта --- это наличие работы слаженного колектива для решения любых поставленных задач.\\
\end{frame}

\begin{frame} \frametitle{Становление команды и этапы формирования}

  \begin{itemize}
  \item Объединение. Характеризуется избытком энтузиазма, связанного с новизной. Люди должны преодолеть внутренние противоречия, переболеть конфликтами прежде, чем сформируется действительно спаянный коллектив.
  \end{itemize}
\end{frame}

\begin{frame} \frametitle{Становление команды и этапы формирования}
 
  \begin{itemize}
  
  \item Разногласия и конфликты. Неизбежные сложности или неудачи порождают конфликты и «поиск виновных». Участники команды методом проб и ошибок вырабатывают наиболее эффективные процессы взаимодействия. Руководителю на этом этапе важно обеспечить открытую коммуникацию в команде.
  \end{itemize}
\end{frame}

\begin{frame} \frametitle{Становление команды и этапы формирования}

  \begin{itemize}
  
  \item Становление. В команде растет доверие, люди начинают замечать в коллегах не только проблемные, но и сильные стороны. Руководитель перестает находиться в состоянии постоянного аврала, работа по построению команды на этом этапе — уже не тушение пожара, а скрупулезный труд по отработке общих норм и правил.
  \end{itemize}

\end{frame}

\begin{frame} \frametitle{Становление команды и этапы формирования}
  \begin{itemize}
  \item Отдача. Команда работает эффективно, высок командный дух, люди хорошо знают друг друга и умеют использовать сильные стороны коллег. Высок уровень доверия. Это лучший период для раскрытия индивидуальных талантов.
  \end{itemize}
\end{frame}

\lecturenotes

Объединение
Этот этап характеризуется, как правило, избытком энтузиазма, связанного с новизной задач. Новизна сама по себе может, как правило, выступать на первых порах значимым мотивирующим фактором. На участников действует так называемый эффект Готорна — прирост энергии и интерес, наполняющий людей, когда они осваивают что-то новое и непривычное.

Люди, объединенные в рабочую группу, имеют различные мотивы и ожидания. Важно понимать, в чем будет выигрыш каждого участника в случае общего успеха проекта, и использовать это знание для сплочения сотрудников. 

Кроме того, в новый коллектив каждый привносит свою «социальную схему», которая представляет собой личные взгляды на то, как должна функционировать команда. Участники группы должны преодолеть внутренние противоречия, пройти через конфликты прежде, чем сформируется действительно спаянный коллектив. На этом этапе многое зависит от лидера: он должен сформировать общекомандное видение проекта. Все участники группы должны четко понимать не только что именно они будут делать, но и почему они будет это делать.

Разногласия и конфликты
Самый сложный и опасный, однако, неизбежный период в становлении команды. Мотивация новизны уже ослабла, а сильные и глубокие внутренние стимулы у команды еще не появились. Каждый участник пытается установить и отстоять свою роль в проекте. На этом этапе возможны соперничество, споры, оборонительная позиция. Неизбежные сложности или неудачи порождают конфликты, «поиск виновных».
Лидеру на этом этапе важно обеспечить открытую коммуникацию в команде — конфликты не следует прятать или разрубать. Споры необходимо разруливать — спокойно, терпеливо и тщательно. Не стоит навязывать свои решения группе, даже если они кажутся очевидными, а следует помогать участникам команды самим приходить к нужным решениям. Не все верные решения могут быть эффективными. Решение будет работать только тогда, когда есть люди, которые хотят его выполнять. Каждый человек мотивирован участвовать в принятии решения по поводу собственных проблем и готов принимать на себя обязательства по выполнению совместно принятых решений.
После «закалки» команды на этапе разногласий и конфликтов в коллективе начинают вырабатываться наиболее оптимальные методы взаимодействия, общения и совместной работы.

Становление
На этапе «Становление» в команде укрепляется доверие, люди начинают замечать в коллегах не только проблемные, но и сильные стороны. На смену битве амбиций приходит продуктивное сотрудничество. Четче становится разделение труда, исчезает дублирование функций. Лидер перестает находиться в состоянии постоянного аврала, работа по построению команды на этом этапе — уже не тушение пожара, а скрупулезный труд по отработке общих норм и правил.
Поскольку, разработка ПО, — корпоративная игра, то первое, что мы должны сделать, это договориться о правилах игры — нормах и регламентах, которые определяют права и ответственность участников команды в проекте. И чем больше и ответственней проект, тем «тяжелей» должна быть технология. Обязательно должны быть сформулированы и внедрены механизмы эффективного пересмотра правил, которые утратили свою актуальность или эффективность. Технология должна изменяться вместе с развитием и ростом проекта, но на каждом этапе она должна быть определена и описана.
Опасность на этом этапе состоит в том, чтобы избежать чрезмерной бюрократии устанавливаемых правил и процедур. Стандарты должны разрабатываться только для решения важных и актуальных проблем, с которыми уже столкнулась команда, а не на все случаи жизни. Регламенты должны давать основные принципы, которыми должны руководствоваться участники проекта, а не подробные правила, которые требуется постоянно соблюдать. Стандарты и регламенты должны облегчать работу участников команды, а не затруднять ее, они — система маяков, которые предостерегают программный проект от того, чтобы он не оказался на мели.

Отдача
Когда наступает этот этап, то, наконец-то, можно приступить к получению дивидендов за потраченные усилия. Команда работает эффективно, высок командный дух, люди хорошо знают друг друга и умеют использовать сильные стороны коллег. Высок уровень доверия. Это лучший период для раскрытия индивидуальных талантов. Люди хотят и могут совершенствоваться, они более всего заинтересованы в профессиональном росте. Растет значение нематериальной мотивации сотрудников, а оценивать и поощрять материально лучше команду в целом.
На этом этапе лидер использует стратегию «делегирования». Руководитель поддерживает на необходимом уровне мотивацию участников команды, следит за качеством их работы. Основное внимание руководителя сосредотачивается на делах из второго квадранта — «точим пилу» Настоящий лидер работает на опережение. Он внимательно следит за изменениями в команде, окружении, целях и задачах проекта — предвидит и избегает риски или снижает их возможные воздействия на проект. Постоянно наблюдает и оценивает эффективность всех процессов, применяемых в проекте. «Что лишнее мы делаем?» «Что можно делать проще?» «Что угрожает проекту?». Работает на сокращение ненужных усилий вместо того, чтобы «стремиться к новым героическим победам». Определяет узкие места и применяет корректирующие действия там, где процессы начинают буксовать или риски слишком велики. Важно: не команда должна приспосабливаться к процессам, а процессы должны подстраиваться под команду по мере ее развития и становления.


\begin{frame} \frametitle{Основные роли лидера}
  \begin{itemize}
  \item Генератор идей
  \item Исследователь ресурсов
  \item Координатор
  \item Мотиватор
  \item Аналитик
  \item Вдохновитель команды
  \item Реализатор
  \item Контролер
  \item Специалист
  \end{itemize}
\end{frame}

\lecturenotes

Для работы с колективом у лидера есть несколько ролей для сплочения и сплаченной работы колектива:
Генератор идей
Оригинальный мыслитель, который дает жизнь новым идеям. Независимый сотрудник с развитым воображением, но подобно остальным людям имеет негативные черты характера — может быть чрезмерно чувствителен к критике. Для успеха генератору идей необходимы конструктивные отношения с руководителем или координатором группы.
Исследователь ресурсов
Так же, как и генератор идей, в состоянии привнести новые идеи в группу, но эти идеи будут заимствованы извне, благодаря широким контактам. Несколько бесцеремонный, гибкий и ищет благоприятные возможности. Обычно разговаривает по телефону или находится где-нибудь на встрече. Не дает развиваться групповой инертности. К отрицательным качествам характера относятся лень, самодовольство и, иногда, требуется кризис или давление обстоятельств, чтобы мотивировать его.
Координатор
Обычно формальный лидер группы. Руководит и направляет группу в сторону достижения целей. Может заранее определить, кто из работников хорош для выполнения необходимых задач. Обычно спокойный, уверенный и распорядительный. Однако иногда склонен к излишнему доминированию, и группа становится продолжением его сильного «Я».
Мотиватор
Энергичный и в состоянии внедрять идеи. Видит мир как проект, который требует внедрения. Обычно уверенный, динамичный, эмоциональный и импульсивный. Мотор группы, но может быть раздражительным, несдержанным, нелюбезным.
Аналитик
Оценивает предложения и занимает позицию наблюдателя за продвижением. Не дает группе двигаться неправильным путем. Осмотрительный, бесстрастный, имеет аналитический склад ума. Может казаться равнодушным, незаинтересованным, иногда становится чрезмерно критичным.	Все ли возможности мы использовали?
Вдохновитель команды
Стремится объединять и вносить гармонию в отношения между членами группы. Занимает позицию понимающего чужие проблемы, стремится помочь и сглаживает конфликты. По натуре человек добрый, стремится налаживать неформальные отношения. Однако бывает нерешительным в сложных или кризисных ситуациях.
Реализатор
Может преобразовать стратегический план в конкретные управленческие задачи, которые доступны для решения. Хороший организатор, методичный и прагматичный. Идентифицируется с группой, лояльный и честный сотрудник. Однако может быть негибким, непреклонным.
Контролер
Отлично умеет создавать отчеты о работе группы. Озабочен точным выполнением взятых обязательств и старается не упускать из виду даже мелких деталей. Заставляет придерживаться точного расписания дел, но может становиться излишне тревожным.	Это дело требует нашего пристального внимания.
Специалист
Профессионал, самостоятелен стремится стать экспертом в своей области. Обладает высокой профессиональной/технической экспертизой и знаниями, гордится своей работой. Приносит вклад только в узкой сфере своей профессиональной экспертизы.

\begin{frame} \frametitle{ Стратегии руководства и их применение}
  \begin{itemize}
  \item Инструкции
  \item Объяснения
  \item Участие 
  \item Делегирование
  \end{itemize}
\end{frame}

\lecturenotes

Инструкции- Директивное управление,жесткое назначение работ,строгий контроль сроков и результатов.Руководитель еще не признан командой как
профессионал, не имеет кредита доверия.

Сочетание директивного и коллективного управления.Объяснение своих решений. Доверие есть, а уверенности в правильности и профессионализме решений нет – надо объяснять.

Участие -Приоритетное коллективное управление, обмен идеями, поддержка инициативы подчиненных. Руководитель признан как профессионал. Но не имеет доверия со стороны команды. Чтобы его завоевать, необходимо активно
привлекать участников коллектива к выработке и принятию решений

Делегирование - Пассивное управление сформировавшегося руководителя. Делегирование полномочий и наблюдение за работой команды.Коллективное
лидерство. Команда становится самоуправляемой. Руководитель отслеживает макропроцессы. Доверие и признание должно быть взаимным. Следует помнить, что руководитель делегирует свои полномочия, но не свою ответственность за
результаты проекта



\section{Тимбилдинг}

\begin{frame} \frametitle{Понятие тимбилдинга}
\begin{block}{}
Тимбилдинг --- созданию благоприятных условий для~работы команды, осуществление мероприятий, нацеленных на сплочение коллектива и его организованности.
\end{block}
\end{frame}



\begin{thebibliography}{99}

%Информацию собирал с нескольких мест на один слайд, поэтому указал списком, а не метками

С. Архипенков Руководство командой разработчиков разработчиков ПО;
Государственный университет  - Высшая школа экономики Факультет Бизнес-информатики Учебное пособие «Лидерство и управление командой»;
\end{thebibliography}

\end{document}

%%% Local Variables: 
%%% mode: TeX-pdf
%%% TeX-master: t
%%% End: 
